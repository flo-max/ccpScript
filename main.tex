\documentclass[a4paper,orivec]{llncs}

\newcommand{\alert}[1]{{\color{red}#1}}

\usepackage{fullpage}
\pagestyle{plain}
\setcounter{tocdepth}{2}

%!TEX root=main.tex

\usepackage{lmodern}
\usepackage[T1]{fontenc}
\usepackage[utf8]{inputenc}

\usepackage{amsfonts,amsmath,amssymb,mathtools}

\usepackage{graphicx}

\usepackage{array}

\usepackage{color,xcolor}
\definecolor{gray}{gray}{0.5}
\definecolor{darkblue}{rgb}{0,0,0.5}
\definecolor{darkgreen}{rgb}{0,0.5,0}
\usepackage[colorlinks=true,linkcolor=darkblue,urlcolor=darkblue,citecolor=darkgreen,pdftitle={Cryptographic Communication Protocols: Key Exchange and Channels}]{hyperref}

\usepackage[strings]{underscore}

\newcommand{\codescalefactor}{0.8}
\newcommand{\tikzscalefactor}{0.8}

\usepackage{nicodemus}
\usepackage{bpmarker}
\usepackage[operators]{cryptocode}
\input{commands}

\bibliographystyle{alpha}

\makeatletter
\renewcommand*\l@author[2]{}
\renewcommand*\l@title[2]{}
\makeatletter
\renewcommand{\contentsname}{}


\title{Cryptographic Communication Protocols:\\Key Exchange and Channels}
\author{Paul Rösler}

\institute{FAU Erlangen-Nürnberg}

\begin{document}


\maketitle
\begin{center}
    \today
\end{center}

\begingroup
\let\clearpage\relax
\tableofcontents
\endgroup

\section{Preliminary Remarks}
This document is not (yet) a full course script.
Instead, it is meant as an additional resource that systematizes the considered primitives, definitions, and constructions.
The author invites readers to submit comments or pull requests via the GitHub Repository \url{https://github.com/roeslpa/ccpScript}.


\section{Notation}

\begin{tabular}{|l|p{13cm}|}\hline
    \textbf{Symbol} & \textbf{Meaning}\\\hline
    $\T$ / $\F$ & Boolean values true and false\\
    $\gets$ / $\to$ & Assigns a constant expression or the output of a deterministic algorithm\\
    $\getsr$ / $\tor$ & Assigns a random value uniformly sampled from a finite set or the output of a probabilistic algorithm\\
    $\PS(X)$ & Power set of set~$X$\\
    $(X)^+$ & Set of non-trivial concatenations of elements from set~$X$\\
    $X[\cdot]\gets x$ & Assigns all entries of array~$X$ with default value~$x$\\
    Invoke $X$ & Executes algorithm $X$\\
    Stop with $x$ & Terminates the running experiment with final output~$x$\\
    Require $x$ & Ends the algorithm, resp.~oracle, if expression~$x=\F$\\\hline
\end{tabular}

\section{Overview}

The goal of this course is to \emph{understand how to analyze the security of real-world protocols}.
\emph{Analyzing} comprises
\begin{enumerate}
    \item Methodically Defining Security
    \item Identifying the Cryptographic Core of Protocols
    \item Systematically Finding Attacks against Protocols
    \item Formally Proving Security of Protocols
\end{enumerate}
This course focuses on the following classes of \emph{real-world protocols} and their main building blocks:
\begin{itemize}
    \item Key Exchange Protocols
    \item Simple Communication Channels
    \item Messaging Protocols
    \item Group Membership Management Protocols
\end{itemize}

\subsection{First Informal Example: Diffie-Hellman Key Exchange}
Let $\DHgr=(\DHg,\DHp)$ be a group of prime order~$\DHp$ with generator~$\DHg$.
Then, the Diffie-Hellman Key Exchange~(DHKE)~\cite{DifHel76} between two users Alice and Bob works as follows:
\begin{enumerate}
    \item Alice samples $x\getsr\ZZ_\DHp^*$, locally stores $x$, and sends $X\gets(\DHg^x\mod\DHp)$ to Bob
    \item Bob samples $y\getsr\ZZ_\DHp^*$, computes $k\gets(X^y\mod\DHp)$, and sends $Y\gets(\DHg^y\mod\DHp)$ to Alice
    \item Alice computes $k\gets(Y^x\mod\DHp)$, and forgets $x$.
\end{enumerate}
Since $k\equiv X^y\equiv\DHg^{xy}\equiv Y^x\equiv k\mod \DHp$, Alice and Bob compute the same shared key.

\begin{figure}
    \centering
    \begin{tabular}{lC{3cm}l}
         \textbf{Alice} & & \textbf{Bob}\\
         $x\getsr\ZZ_\DHp^*$ & &\\
         $X\gets\DHg^x\mod\DHp$ & $\xrightarrow{X}$ & $y\getsr\ZZ_\DHp^*$\\
         & $\xleftarrow{Y}$ & $Y\gets \DHg^y\mod\DHp$\\
         $k\gets Y^x\mod\DHp$ & & $k\gets X^y\mod\DHp$\\
    \end{tabular}
    \caption{Diffie-Hellman Key Exchange.}
    \label{fig:dhke}
\end{figure}

\paragraph{Security of DHKE}
To prove security of the DHKE protocol, we use the \emph{Decisional Diffie-Hellman}~(DDH) problem.
The advantage of an adversary~$\advA$ against the DDH problem in group~$\DHgr$ is defined as:
\begin{align*}
    \Adv_\DHgr^\ddh(\advA)\coloneqq|&\Pr[\advA(\DHg,\DHp,\DHg^x,\DHg^y,\DHg^{xy})=1\mid (x,y)\getsr(\ZZ_\DHp^*)^2]\\
    &-\Pr[\advA(\DHg,\DHp,\DHg^x,\DHg^y,\DHg^z)=1\mid (x,y,z)\getsr(\ZZ_\DHp^*)^3]|\text{.}
\end{align*}
Intuitively, adversary~$\advA$ plays a game in which it sees both Diffie-Hellman shares $\DHg^x$ and $\DHg^y$ as well as either the real key~$\DHg^{xy}$ or a random group element~$\DHg^z$.
To win this game, the adversary has to \emph{decide} which of the two views they saw.
The advantage $\Adv_\DHgr^\ddh(\advA)$ denotes the probability of $\advA$ winning this game beyond pure guessing.

We define the \emph{Computational DH}~(CDH) and the \emph{Discrete Logarithm}~(DLog) problems for prime-order groups in Section~\ref{sec:asym:assumptions:dh}.

Based on the hardness of the best known attacks against the DDH problem from the long history of work in number theory, we assume that the DDH problem is \emph{hard} for security parameter~$\secp$.
This means that $\Adv_\DHgr^\ddh(\advA)$ is \emph{negligible} for all adversaries~$\advA$ that can be specified as \emph{probabilistic polynomial-time Turing machines} (PPT).

\paragraph{Simplifications}
For simplicity and clarity, we refrain from considering concrete group generation algorithms and we do not elaborate on hardness analyses of the DDH problem.
Furthermore, instead of treating security \emph{asymptotically} with respect to a security parameter~$\secp$, using the definitions of ``negligible'' and ``PPT adversaries'', we consider security \emph{concretely} in this document.
This means that all our reductionist security proofs provide concrete relations between the security of the analyzed protocols and the hardness of the underlying assumptions.

\paragraph{Real-World Remarks}
In practice, Elliptic Curves are often used to instantiate prime-order groups~$\DHgr$.
While the DHKE is one of the most widely used key exchange protocols in the real world, we want to mention that Shor's algorithm~\cite{FOCS:Shor94} solves the DDH problem in polynomial time using a quantum computer.
Thus, we only use the DHKE as a simple, introductory real-world example.

\paragraph{Intuitive Security Definition: Passive Adversaries}
A passive adversary~$\advA$ against the DHKE protocol is \emph{capable} of seeing the transcript of the interaction between Alice and Bob: $X=\DHg^x$ and $Y=\DHg^y$.
The \emph{goal} of the adversary is to derive information about the exchanged key~$k=\DHg^{xy}$ from that seen transcript.
To capture this goal, we challenge~$\advA$ eventually on key~$k$:
$\advA$ either sees the real key~$k_0=k=\DHg^{xy}$ or a randomly sampled element from the key space~$k_1\getsr\ksp$, where $k_1\gets\DHg^z$ and $z\getsr\ZZ_\DHp^*$.
If the adversary can distinguish these two cases, this is considered a \emph{successful} attack against the key exchange protocol.
In contrast, if we can prove that distinguishing the real key from a random key is hard, then we can treat the real key as if it was sampled randomly from a uniform distribution;
more intuitively, then we know that the adversary has no information about the real key, even when observing the interaction between Alice and Bob.

We emphasize that the above intuitive description of the adversarial capabilities, the attack goal, and the success criteria should only sketch the expected security requirements for the DHKE protocol.
As we will see, \emph{good} security definitions should not be specified with a concrete protocol design in mind.

\paragraph{Sketch of Security Proof}
The above described passive adversary~$\advA$ against DHKE obtains inputs $(X,Y)$ and $k_b$, where $X=\DHg^x$, $Y=\DHg^y$, $k_0=k=\DHg^{xy}$, $k_1=\DHg^z$, $(x,y,z)\in\ZZ_\DHp^*$, and $b\in\BB$.
If $\advA$, upon processing these inputs, outputs $b'$ such that $b=b'$, $\advA$ breaks the required security goal.

\begin{figure}
    \centering
    \input{figures/dhke_reduction}
    \caption{Sketch of reduction from Diffie-Hellman key exchange to DDH problem.}
    \label{fig:dhke:reduction}
\end{figure}

The primary proof method in this course is based on reductions via sequences of games~\cite{EPRINT:Shoup04} that we will specify in pseudo-code~\cite{EC:BelRog06}.
However, for the proof sketch of our simple example in which we reduce the security of DHKE to the hardness of the DDH problem, the schematic overview of reduction $\advB^\ddh$ in Figure~\ref{fig:dhke:reduction} suffices:
The reduction uses its DDH-challenge~$Z$ as the challenge key~$k$.
Once the DHKE-adversary~$\advA$ solves this challenge by successfully guessing bit~$b'$, the reduction can use this guess to break the DDH-challenge via bit~$b''$.
A formal proof remains a task for the interested reader.

\subsection{Second Example: Public-Key Encryption (PKE)}

\paragraph{Syntax}
A public-key encryption scheme $\PKE=(\PKEgen,\PKEenc,\PKEdec)$ is a tuple of three algorithms with encryption key space~$\eksp$, decryption key space~$\dksp$, message space~$\msp$, and ciphertext space~$\csp$:

\begin{itemize}
    \item $\PKEgen: \emptyset \tor \dksp\times\eksp$
    \item $\PKEenc: \eksp\times\msp \tor \csp$
    \item $\PKEdec: \dksp\times\csp \to \msp$
\end{itemize}

\paragraph{Correctness}
A public-key encryption scheme $\PKE$ is correct if
\[
\Pr[\PKEdec(\dk,\PKEenc(\ek,m))=m\mid (\dk,\ek)\getsr\PKEgen,m\getsr\msp]=1\text{.}
\]

\paragraph{Passive Security}
We begin with a notion of passive security for PKE that is called \emph{Indistinguishability of Ciphertexts under Chosen Plaintext Attacks} (IND-CPA).
This models that adversaries only eavesdrop victims' ciphertexts for adversarially chosen messages---hence, \emph{chosen plaintext attacks}.
However, the adversary never observes the decryption of ciphertexts that were not created by the victims themselves.
The goal of the adversary is to decide which of two possible messages was encrypted in a challenge ciphertext---hence, \emph{indistinguishability of ciphertext}.

We introduce this passive notion of security mainly for didactic reasons.
Yet, we want to mention that the Fujisaki-Okamoto transform~\cite{C:FujOka99} lifts passively secure PKE schemes to actively secure ones.
Hence, this notion is indeed relevant in practice.

The advantage of an adversary~$\advA$ against public-key encryption scheme $\PKE$ in game $\INDCPA$ from Figure~\ref{fig:pke:ind:cpa} is defined as:
\[
\Adv_\PKE^\indcpa(\advA)\coloneqq\left|\Pr[\INDCPA_{\PKE}^0(\advA)=1]-\Pr[\INDCPA_{\PKE}^1(\advA)=1]\right|\text{.}
\]

\begin{figure}[!ht]
    \centering
    \nicoresetlinenr%
    \fbox{%
        \scalebox{\codescalefactor}{%
            \input{figures/pke_ind_cpa}%
        }%
    }
    \caption{%
        Games $\INDCPA$ for public-key encryption scheme~$\PKE$.
    }
    \label{fig:pke:ind:cpa}
\end{figure}

\paragraph{Active Security}
The ability to actively manipulate ciphertexts and observe the victims' reaction on the decryption is modeled by providing a decryption oracle.
This additional oracle returns decrypted messages for all ciphertext queries except for those that correspond to challenge ciphertexts.
If also decryptions of challenge ciphertexts were given to the adversary, the adversary could trivially win the game for any functional PKE construction.
Therefore, we call such attack strategies \emph{trivial attacks} or \emph{trivial winning conditions}.

\begin{figure}[!ht]
    \centering
    \nicoresetlinenr%
    \fbox{%
        \scalebox{\codescalefactor}{%
            \input{figures/pke_ind}%
        }%
    }
    \caption{%
        Games $\INDCCA$ for public-key encryption scheme~$\PKE$.
    }
    \label{fig:pke:ind}
\end{figure}

The advantage of an adversary~$\advA$ against public-key encryption scheme $\PKE$ in game $\INDCCA$ from Figure~\ref{fig:pke:ind} is defined as:
\[
\Adv_\PKE^\indcca(\advA)\coloneqq\left|\Pr[\INDCCA_{\PKE}^0(\advA)=1]-\Pr[\INDCCA_{\PKE}^1(\advA)=1]\right|\text{.}
\]

\paragraph{ElGamal Encryption}
As shown in Figure~\ref{fig:pke:const:elgamal}, the ElGamal public-key encryption scheme~\cite{ElGamal85} at its core uses the Diffie-Hellman key exchange.
Intuitively, the ElGamal PKE key pair~$(\dk,\ek)$ is Alice's key pair in the DHKE.
For ElGamal encryption, one samples a fresh, ephemeral DHKE key pair, which corresponds to Bob's key in the DHKE.
Using Alice's DHKE public key and Bob's DHKE secret key, one computes the shared key and multiplies the message to that shared key.
On ElGamal decryption, the same shared key is computed and, by multiplying with its inverse, the original, encrypted message is derived.

\begin{figure}[!ht]
    \centering
    \nicoresetlinenr%
    \fbox{%
        \scalebox{\codescalefactor}{%
            \input{figures/pke_const_elgamal}%
        }%
    }
    \caption{%
        ElGamal public-key encryption scheme~$\PKE$~\cite{ElGamal85}.
    }
    \label{fig:pke:const:elgamal}
\end{figure}


\section{Game-Based Definitions}

\subsection{Syntax}

\subsection{Correctness}

\subsection{Adversarial Capabilities}

\subsubsection{Typical Capabilities}

\paragraph{Chosen-Plaintext Attacks}

\paragraph{Chosen-Ciphertext Attacks}

\paragraph{Replayable Chosen-Ciphertext Attacks}

\paragraph{Exposure of Secrets}

\paragraph{Compromised Randomness}

\subsection{Adversarial Goal}

\subsubsection{Typical Goals}

\paragraph{One-Way Security}

\paragraph{Indistinguishability of Ciphertexts}

\subsection{Trivial Winning Strategies}


\section{Symmetric Primitives}


\subsection{Pseudo-Random Generator}


\paragraph{Syntax}
A pseudo-random generator $\PRG=\PRGf$ is an algorithm with key space~$\ksp$ and output space~$\rsp$:

\begin{itemize}
    \item $\PRGf: \ksp \to \rsp$
\end{itemize}

\subsection{Pseudo-Random Function}

\paragraph{Syntax}
A pseudo-random function $\PRF=\PRFf$ is an algorithm with key space~$\ksp$, input space~$\msp$ and output space~$\rsp$:

\begin{itemize}
    \item $\PRGf: \ksp\times\msp \to \rsp$
\end{itemize}

\subsection{Message Authentication Code (MAC)}

\paragraph{Syntax}
A message authentication code $\MAC=(\MACgen,\MACtag,\MACvfy)$ is a tuple of three algorithms with key space~$\ksp$, message space~$\msp$, and tag space~$\tsp$:

\begin{itemize}
    \item $\MACgen: \emptyset \tor \ksp$
    \item $\MACtag: \ksp\times\msp \tor \tsp$
    \item $\MACvfy: \ksp\times\msp\times\tsp \to \Bool$
\end{itemize}

\paragraph{Correctness}
A message authentication code $\MAC$ is correct if
\[
\Pr[\MACvfy(k,m,\MACtag(k,m))=\T\mid k\getsr\MACgen,m\getsr\msp]=1\text{.}
\]

\paragraph{Security: Strong Existential Unforgeability under Chosen-Message Attacks}
The advantage of an adversary~$\advA$ against message authentication code $\MAC$ in game $\SUFCMA$ from Figure~\ref{fig:mac:suf} is defined as:
\[
\Adv_\MAC^\sufcma(\advA)\coloneqq\Pr[\SUFCMA_{\MAC}(\advA)=1]\text{.}
\]

\begin{figure}[!ht]
    \centering
    \nicoresetlinenr%
    \fbox{%
        \scalebox{\codescalefactor}{%
            \input{figures/mac_suf}%
        }%
    }
    \caption{%
        Game $\SUFCMA$ for message authentication code~$\MAC$.
    }
    \label{fig:mac:suf}
\end{figure}


\subsection{Symmetric Encryption (SE)}

\subsubsection{Probabilistic SE}

\paragraph{Syntax}
A probabilistic symmetric encryption scheme $\SYE=(\SYEgen,\SYEenc,\SYEdec)$ is a tuple of three algorithms with key space~$\ksp$, message space~$\msp$, and ciphertext space~$\csp$:

\begin{itemize}
    \item $\SYEgen: \emptyset \tor \ksp$
    \item $\SYEenc: \ksp\times\msp \tor \csp$
    \item $\SYEdec: \ksp\times\csp \to \msp$
\end{itemize}

\paragraph{Correctness}
A symmetric encryption scheme $\SYE$ is correct if 
\[
\Pr[\SYEdec(k,\SYEenc(k,m))=m\mid k\getsr\SYEgen,m\getsr\msp]=1\text{.}
\]
Equivalently, a symmetric encryption scheme $\SYE$ is correct if $\Pr[\CORR_{\SYE}(\advA)=0]=1$ for all adversaries~$\advA$, where game~$\CORR$ is defined in Figure~\ref{fig:sym:enc:corr:prob}.

\begin{figure}[!ht]
    \centering
    \nicoresetlinenr%
    \fbox{%
        \scalebox{\codescalefactor}{%
            \input{figures/sym_enc_corr_prob}%
        }%
    }
    \caption{%
        Game $\CORR$ for probabilistic encryption scheme~$\SYE$.
    }
    \label{fig:sym:enc:corr:prob}
\end{figure}

\paragraph{Security: One-Wayness under Chosen-Ciphertext Attacks}
The advantage of an adversary~$\advA$ against symmetric encryption scheme $\SYE$ in game $\OWCCA$ from Figure~\ref{fig:sym:enc:ow:prob} is defined as:
\[
\Adv_\SYE^\owcca(\advA)\coloneqq\Pr[\OWCCA_{\SYE}(\advA)=1]\text{.}
\]

\begin{figure}[!ht]
    \centering
    \nicoresetlinenr%
    \fbox{%
        \scalebox{\codescalefactor}{%
            \input{figures/sym_enc_ow_prob}%
        }%
    }
    \caption{%
        Game $\OWCCA$ for probabilistic symmetric encryption scheme~$\SYE$.
    }
    \label{fig:sym:enc:ow:prob}
\end{figure}

\paragraph{Security: Indistinguishability under Chosen-Ciphertext Attacks}
The advantage of an adversary~$\advA$ against symmetric encryption scheme $\SYE$ in game $\INDCCA$ from Figure~\ref{fig:sym:enc:ind:prob} is defined as:
\[
\Adv_\SYE^\indcca(\advA)\coloneqq\left|\Pr[\INDCCA_{\SYE}^0(\advA)=1]-\Pr[\INDCCA_{\SYE}^1(\advA)=1]\right|\text{.}
\]

\begin{figure}[!ht]
    \centering
    \nicoresetlinenr%
    \fbox{%
        \scalebox{\codescalefactor}{%
            \input{figures/sym_enc_ind_prob}%
        }%
    }
    \caption{%
        Games $\INDCCA$ for probabilistic symmetric encryption scheme~$\SYE$.
    }
    \label{fig:sym:enc:ind:prob}
\end{figure}



\section{Asymmetric Primitives}
\label{sec:asym}

\subsection{Assumptions}
\label{sec:asym:assumptions}

\subsubsection{Diffie-Hellman (DH)}
\label{sec:asym:assumptions:dh}
Let $\DHgr=(\DHg,\DHp)$ be a group of prime order~$\DHp$ with generator~$\DHg$.

\paragraph{Discrete Logarithm}
The advantage of an adversary~$\advA$ against the discrete logarithm problem in group~$\DHgr$ is defined as:
\[
\Adv_\DHgr^\dlp(\advA)\coloneqq\Pr[\advA(\DHg,\DHp,\DHg^x)=x\mid x\getsr\ZZ_\DHp^*]\text{.}
\]


\paragraph{Computational DH}
The advantage of an adversary~$\advA$ against the computational Diffie-Hellman problem in group~$\DHgr$ is defined as:
\[
\Adv_\DHgr^\cdh(\advA)\coloneqq\Pr[\advA(\DHg,\DHp,\DHg^x,\DHg^y)=,\DHg^{xy}\mid (x,y)\getsr(\ZZ_\DHp^*)^2]\text{.}
\]

\paragraph{Decisional DH}
The advantage of an adversary~$\advA$ against the decisional Diffie-Hellman problem in group~$\DHgr$ is defined as:
\begin{align*}
    \Adv_\DHgr^\ddh(\advA)\coloneqq|&\Pr[\advA(\DHg,\DHp,\DHg^x,\DHg^y,\DHg^{xy})=1\mid (x,y)\getsr(\ZZ_\DHp^*)^2]\\
    &-\Pr[\advA(\DHg,\DHp,\DHg^x,\DHg^y,\DHg^z)=1\mid (x,y,z)\getsr(\ZZ_\DHp^*)^3]|\text{.}
\end{align*}

\subsection{Digital Signature}

\paragraph{Syntax}
A digital signature scheme $\SIG=(\SIGgen,\SIGsig,\SIGvfy)$ is a tuple of three algorithms with signing key space~$\sksp$, verification key space~$\vksp$, message space~$\msp$, and signature space~$\sigsp$:

\begin{itemize}
    \item $\SIGgen: \emptyset \tor \sksp\times\vksp$
    \item $\SIGsig: \sksp\times\msp \tor \sigsp$
    \item $\SIGvfy: \vksp\times\msp\times\sigsp \to \Bool$
\end{itemize}

\paragraph{Correctness}
A digital signature scheme $\SIG$ is correct if
\[
\Pr[\SIGvfy(\vk,m,\SIGsig(\sk,m))=\T\mid (\sk,\vk)\getsr\SIGgen,m\getsr\msp]=1\text{.}
\]

\paragraph{Security}
The advantage of an adversary~$\advA$ against digital signature scheme $\SIG$ in game $\SUFCMA$ from Figure~\ref{fig:sig:suf} is defined as:
\[
\Adv_\SIG^\sufcma(\advA)\coloneqq\Pr[\SUFCMA_{\SIG}(\advA)=1]\text{.}
\]

\begin{figure}[!ht]
    \centering
    \nicoresetlinenr%
    \fbox{%
        \scalebox{\codescalefactor}{%
            \input{figures/sig_suf}%
        }%
    }
    \caption{%
        Game $\SUFCMA$ for digital signature scheme~$\SIG$.
    }
    \label{fig:sig:suf}
\end{figure}

\subsection{Key Encapsulation Mechanism (KEM)}

\paragraph{Syntax}
A key encapsulation mechanism $\KEM=(\KEMgen,\KEMenc,\KEMdec)$ is a tuple of three algorithms with encapsulation key space~$\eksp$, decapsulation key space~$\dksp$, ciphertext space~$\csp$, and symmetric key space~$\ksp$:

\begin{itemize}
    \item $\KEMgen: \emptyset \tor \dksp\times\eksp$
    \item $\KEMenc: \eksp \tor \ksp\times\csp$
    \item $\KEMdec: \dksp\times\csp \to \ksp$
\end{itemize}

\paragraph{Correctness}
A key encapsulation mechanism $\KEM$ is correct if
\[
\Pr[\KEMdec(\dk,c)=k\mid (\dk,\ek)\getsr\KEMgen,(k,c)\getsr\KEMenc(\ek))=m]=1\text{.}
\]
Equivalently, a key encapsulation mechanism $\KEM$ is correct if $\Pr[\CORR_{\KEM}(\advA)=0]=1$ for all adversaries~$\advA$, where game~$\CORR$ is defined in Figure~\ref{fig:kem:corr}.

\begin{figure}[!ht]
    \centering
    \nicoresetlinenr%
    \fbox{%
        \scalebox{\codescalefactor}{%
            \input{figures/kem_corr}%
        }%
    }
    \caption{%
        Game $\CORR$ for key encapsulation mechanism~$\KEM$.
    }
    \label{fig:kem:corr}
\end{figure}

\paragraph{Security}
The advantage of an adversary~$\advA$ against key encapsulation mechanism $\KEM$ in game $\INDCCA$ from Figure~\ref{fig:kem:ind} is defined as:
\[
\Adv_\KEM^\indcca(\advA)\coloneqq\left|\Pr[\INDCCA_{\KEM}^0(\advA)=1]-\Pr[\INDCCA_{\KEM}^1(\advA)=1]\right|\text{.}
\]

\begin{figure}[!ht]
    \centering
    \nicoresetlinenr%
    \fbox{%
        \scalebox{\codescalefactor}{%
            \input{figures/kem_ind}%
        }%
    }
    \caption{%
        Games $\INDCCA$ for key encapsulation mechanism~$\KEM$.
    }
    \label{fig:kem:ind}
\end{figure}

\paragraph{Construction: ElGamal KEM}


\subsection{Identity-Based KEM}

\paragraph{Syntax}
An identity-based KEM $\IKEM=(\IKEMgen,\IKEMenc,\IKEMdel,\IKEMdec)$ is a tuple of four algorithms with encapsulation key space~$\eksp$, decapsulation key space~$\dksp$, identity space~$\idsp$, ciphertext space~$\csp$, and symmetric key space~$\ksp$:

\begin{itemize}
    \item $\IKEMgen: \emptyset \tor \dksp\times\eksp$
    \item $\IKEMenc: \eksp\times(\idsp)^+ \tor \ksp\times\csp$
    \item $\IKEMdel: \dksp\times\idsp \tor \dksp$
    \item $\IKEMdec: \dksp\times\csp \to \ksp$
\end{itemize}


\section{Stateful Primitives}

\subsection{Forward-Secure KEM}

\paragraph{Syntax}
A forward-secure KEM $\FKEM=(\FKEMgen,\FKEMenc,\FKEMdec)$ is a tuple of three algorithms with encapsulation key space~$\eksp$, decapsulation key space~$\dksp$, ciphertext space~$\csp$, and symmetric key space~$\ksp$:

\begin{itemize}
    \item $\FKEMgen: \emptyset \tor \dksp\times\eksp$
    \item $\FKEMenc: \eksp \tor \ksp\times\csp$
    \item $\FKEMdec: \dksp\times\csp \to \dksp\times\ksp$
\end{itemize}

\paragraph{Correctness}
A forward-secure KEM $\FKEM$ is correct if $\Pr[\CORR_{\FKEM}(\advA)=0]=1$ for all adversaries~$\advA$, where game~$\CORR$ is defined in Figure~\ref{fig:fkem:corr}.

\begin{figure}[!ht]
    \centering
    \nicoresetlinenr%
    \fbox{%
        \scalebox{\codescalefactor}{%
            \input{figures/fkem_corr}%
        }%
    }
    \caption{%
        Game $\CORR$ for forward-secure KEM~$\FKEM$.
    }
    \label{fig:fkem:corr}
\end{figure}

\paragraph{Security}
The advantage of an adversary~$\advA$ against forward-secure KEM $\FKEM$ in game $\INDCCA$ from Figure~\ref{fig:fkem:ind} is defined as:
\[
\Adv_\FKEM^\indcca(\advA)\coloneqq\left|\Pr[\INDCCA_{\FKEM}^0(\advA)=1]-\Pr[\INDCCA_{\FKEM}^1(\advA)=1]\right|\text{.}
\]

\begin{figure}[!ht]
    \centering
    \nicoresetlinenr%
    \fbox{%
        \scalebox{\codescalefactor}{%
            \input{figures/fkem_ind}%
        }%
    }
    \caption{%
        Games $\INDCCA$ for forward-secure KEM~$\FKEM$.
    }
    \label{fig:fkem:ind}
\end{figure}

\paragraph{Construction}


\subsection{Updatable KEM}

\paragraph{Syntax}
An updatable KEM $\UKEM=(\UKEMgen,\UKEMenc,\UKEMdec)$ is a tuple of three algorithms with encapsulation key space~$\eksp$, decapsulation key space~$\dksp$, ciphertext space~$\csp$, and symmetric key space~$\ksp$:

\begin{itemize}
    \item $\UKEMgen: \emptyset \tor \dksp\times\eksp$
    \item $\UKEMenc: \eksp \tor \eksp\times\ksp\times\csp$
    \item $\UKEMdec: \dksp\times\csp \to \dksp\times\ksp$
\end{itemize}

\paragraph{Construction}


\subsection{Key Exchange}

\subsubsection{Two-Pass Key Exchange}

\paragraph{Syntax}
A two-pass key exchange $\TKE=(\TKEsnd,\TKErsp,\TKErcv)$ is a tuple of three algorithms with state space~$\stsp$, initial ciphertext space~$\csp_I$, response ciphertext space~$\csp_R$, and symmetric key space~$\ksp$:

\begin{itemize}
    \item $\TKEsnd: \emptyset \tor \stsp\times\csp_I$
    \item $\TKErsp: \csp_I \tor \ksp\times\csp_R$
    \item $\TKErcv: \stsp\times\csp_R \to \ksp$
\end{itemize}

\paragraph{Construction: DH Key Exchange}

\paragraph{Construction: X3DH}

\subsubsection{Authenticated Key Exchange}

\paragraph{Syntax}
An authenticated key exchange $\AKE=(\AKEgen,\allowbreak\AKEsnd,\allowbreak\AKErsp_R,\allowbreak\AKErsp_I,\allowbreak\AKErcv)$ is a tuple of five algorithms with secret key space~$\sksp$, public key space~$\pksp$, state space~$\stsp$, ciphertext space~$\csp$, and symmetric key space~$\ksp$:

\begin{itemize}
    \item $\AKEgen: \emptyset \tor \sksp\times\pksp$
    \item $\AKEsnd: \sksp\times\pksp \tor \stsp\times\csp$
    \item $\AKErsp_R: \sksp\times\csp \tor \stsp\times\csp$
    \item $\AKErsp_I: \sksp\times\stsp\times\csp \tor \ksp\times\csp$
    \item $\AKErcv: \sksp\times\stsp\times\csp \to \pksp\times\ksp$
\end{itemize}

\paragraph{Construction: Signature-Based Authentication}

\begin{figure}[!ht]
      \centering
      \fbox{\scalebox{\codescalefactor}{\procedure[]{}{
    \textbf{Alice} \>\> \textbf{Bob} \\[0.1\baselineskip][\hline]\\[0.1\baselineskip]
    a \sample \mathbb{Z}_p^* \\
    \> \sendmessageright*{g^a} \\
    \>\> b \sample \mathbb{Z}_p^* \\
    \>\> \sigma_B \gets sig(sk_B, g^a || g^b) \\
    \> \sendmessageleft*{g^b, \sigma_B} \\
    \sigma_A \gets sig(sk_A, g^a || g^b || \sigma_B) \\
    \>\sendmessageright*{\sigma_A} \\
}}}
      \caption{Construction for Signature-based Authentication.}
      \label{fig:sig_based_auth}
\end{figure}

\paragraph{Construction: KEM-Based Authentication}

\begin{figure}[!ht]
     \centering
     \fbox{\scalebox{\codescalefactor}{\procedure[colspace=-1cm]{}{
        \textbf{Alice} \>\> \textbf{Bob} \\[0.1\baselineskip][\hline]
        \> \\[-0.5\baselineskip]
        K_1, c_1 \sample enc(pk_B) \\
        (sk', pk') \sample gen  \\
        \>\sendmessageright*{c_1, pk'} \\
        \>\> K_1 \gets dec(sk_B, c_1) \\
        \>\>(K_2, c_2) \sample enc(pk_A) \\
        \>\>(K_3, c_3) \sample enc(pk') \\
        \>\> (\tau_B, K_B) \gets PRF(K_1, tag=(c_1, pk', c_2, c_3)) \\
        \> \sendmessageleft*{c_2, c_3, \tau_B} \\
        K_2 \gets dec(sk_A, c_2) \\
        K_3 \gets dec(sk', c_3) \\
        (\tau_B, K_B) \gets PRF(K_1, tag=(c_1, pk', c_2, c_3)) \\
        (K, \tau_A) \gets PRF(K_B, K_2, K_3) 
        \>\> (K, \tau_A) \gets PRF(K_B, K_2, K_3) \\
        \>\sendmessageright*{\tau_A} \\
}}}
     \caption{Construction for KEM-based Authentication.}
     \label{fig:kem_based_auth}
\end{figure}

\paragraph{Construction: TLS}

\paragraph{Construction: Noise Framework}


\subsection{Ratcheted Key Exchange (RKE)}

\subsubsection{Unidirectional RKE}

\paragraph{Syntax}
A unidirectional key exchange $\URKE=(\URKEinit,\URKEsnd,\URKErcv)$ is a tuple of three algorithms with state space~$\stsp$, ciphertext space~$\csp$, and symmetric key space~$\ksp$:

\begin{itemize}
    \item $\URKEinit: \emptyset \tor \stsp\times\stsp$
    \item $\URKEsnd: \stsp \tor \stsp\times\ksp\times\csp$
    \item $\URKErcv: \stsp\times\csp \to \ksp$
\end{itemize}

\begin{figure}[!ht]
    \centering
    \fbox{\scalebox{\codescalefactor}{\procedure{}{
    \textbf{Alice} \>\> \textbf{Bob} \\[0.1\baselineskip][\hline]\\[0.1\baselineskip]
    \> K \leftarrow \fbox{\textit{Initial Key Exchange}} \rightarrow K \> \pclb
    \pcintertext[dotted]{}
    st_A, \_ \gets init(K) \>\> \_, st_B \gets init(K) \\
    (st_A, k, c) \sample snd(st_A) \\
    \>\sendmessageright*{c} \\
    \>\> st_B, k \gets rcv(st_B, c) \pclb
}}}
    \caption{Construction for Unidirectional Ratcheted Key Exchange}
    \label{fig:rke:unidirectional}
\end{figure}

\paragraph{Construction}

\subsubsection{Sesquidirectional RKE}

\subsubsection{Bidirectional RKE}

\paragraph{Construction: Double Ratchet}

\subsubsection{Group RKE}

\paragraph{Construction: Sender Key Mechanism}

\paragraph{Construction: Tree-Based DH}

\paragraph{Construction: Tree KEM}

\bibliography{cryptobib/abbrev3,cryptobib/crypto}

\end{document}